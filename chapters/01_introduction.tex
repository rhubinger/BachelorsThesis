% !TeX root = ../main.tex
% Add the above to each chapter to make compiling the PDF easier in some editors.

\chapter{Introduction}\label{chapter:introduction}

LAIK is a library for data management in the HPC environment.
Its focus lies on providing lightweight fault tolerance and load balancing mechanisms.
The LAIK library sits between the application and the communication library used to migrate the data.
Multiple backends are supported and application programmers can implement new backends for their own communication libraries.
Shared memory is a promising candidate for such a new backend as it is a very efficient method for inter process communication.
The only disadvantage of shared memory is that it is limited to one node as different nodes can not share memory.

Our work presents the design and implementation of two shared memory based backends.
We will implement both a standalone backend capable of running LAIK applications on a single node as well as a secondary backend which will provide shared memory based data migration for other LAIK backends.
We will first introduce the core structure of the LAIK library and the used IPC mechanisms in \autoref{chapter:background}.
The design of our library will be presented in \autoref{chapter:design}.
There we will elaborate on the initialization of the backend, the data transport over shared memory and the differences between the two different backend versions.
In \autoref{chapter:implementation} we will explain how the previously described designs were implemented.
We will explain the structure of our backend and how a backend can use the functionality of our secondary backend version.
We will then go into detail about our implementation of the initialization process and the data transport.
Our implemented design will then be tested against other LAIK backends in \autoref{chapter:performance_analysis}.
We conclude our work by discussing possible optimizations for future work in \autoref{chapter:summary_and_outlook}

